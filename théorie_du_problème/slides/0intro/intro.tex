\begin{minipage}{0.78\linewidth}
\vspace{0.5cm}
\footnotesize\textbf{Université catholique de Louvain}\\
Ecole polytechnique\\
Année académique 2013-2014\\
\end{minipage}
\begin{minipage}{0.1\linewidth}
\begin{flushright}
\includegraphics[width=2cm]{images/logo_EPL.png}
\end{flushright}
\end{minipage}

\vspace{1.5cm}
\par
\hrule height 1.5pt
\par
\vspace{0.5cm}

{\centering
\Large \textbf{Détection de cycle dans un graphe dirigé}\\
\normalsize  Conventions de représentations et théorie du problème\\
\vspace{0.3cm}
\footnotesize Groupe 9\\
LINGI1122 - Méthodes de conception de programmes\\
Titulaire : José Vander Meulen\\}

\vspace{0.5cm}
\par
\hrule height 1.5pt
\par
\vspace{0.5cm}

%\begin{figure}[!h]
%	\begin{center}
%		\includegraphics[width=2.8cm]{images/logo_EPL.png}
%	\end{center}
%\end{figure}


\begin{frame}{Table des matières}
\begin{enumerate}
  \item Introduction
  \item Description générale
  \item Description détaillée
  \begin{enumerate}
    \item Bobine émettrice
    \item Bobine portable
    	\begin{enumerate}
		 \item Oscillateur RC
		 \item Signal codé
		 \item Modulateur
		 \item Dimensionnement de la bobine
		 \end{enumerate}
    \item Bobine réceptrice
    	\begin{enumerate}
		 \item Amplificateur
		 \item Détecteur de crêtes
		 \item Capacité
		 \item Comparateur
		 \item Décodeur
		 \end{enumerate}
  \end{enumerate}
  \item Tests Paramétriques
  \item Conclusion
\end{enumerate}
\end{frame}
