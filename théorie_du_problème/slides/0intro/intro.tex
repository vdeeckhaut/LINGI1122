\begin{minipage}{0.78\linewidth}
\vspace{0.5cm}
\footnotesize\textbf{Université catholique de Louvain}\\
Ecole polytechnique\\
Année académique 2013-2014\\
\end{minipage}
\begin{minipage}{0.1\linewidth}
\begin{flushright}
\includegraphics[width=2cm]{images/logo_EPL.png}
\end{flushright}
\end{minipage}

\vspace{1.5cm}
\par
\hrule height 1.5pt
\par
\vspace{0.5cm}

{\centering
\Large \textbf{Détection de cycle dans un graphe dirigé}\\
\normalsize  Conventions de représentations et théorie du problème\\
\vspace{0.3cm}
\footnotesize Groupe 9\\
LINGI1122 - Méthodes de conception de programmes\\
Titulaire : José Vander Meulen\\}

\vspace{0.5cm}
\par
\hrule height 1.5pt
\par
\vspace{0.5cm}

%\begin{figure}[!h]
%	\begin{center}
%		\includegraphics[width=2.8cm]{images/logo_EPL.png}
%	\end{center}
%\end{figure}


\begin{frame}{Table des matières}
\begin{enumerate}
  \item Contexte
  \item Conventions de représentation
  \item Théorie du problème
\end{enumerate}
\end{frame}

\begin{frame}{Contexte}
\textbf{Problème}: étant donné un graphe dirigé, déterminer si celui-ci contient ou non un cycle.\\
\vspace{0.5cm}
\textbf{Algorithme}: supprimer tous les nœuds qui n'ont pas d'arête entrante, ainsi que les arêtes dont ces nœuds sont l'origine. En répétant cette opération, deux cas peuvent survenir: 
\begin{enumerate}
\item plus de nœud disponible $\rightarrow$ pas de cycle;
\item il ne reste que des nœuds avec au moins une arête entrante $\rightarrow$ au moins un cycle dans le graphe.
\end{enumerate}
\end{frame}
