\section{Théorie du problème}

\subsection{Définitions}
Un \textbf{graphe} est un triplet $(V,E,\psi)$ où:
\begin{itemize}
\item $V$ est un ensemble dont les éléments sont appelés sommets ou nœuds;
\item $E$ est un ensemble dont les éléments sont appelés arêtes;
\item $\psi$ est une fonction, dite fonction d'incidence, qui associe à chaque arête un sommet ou une paire de sommets.
\end{itemize}
On peut représenter un graphe par un diagramme. Plusieurs diagrammes peuvent représenter le même graphe.\\

Un \textbf{graphe dirigé}, ou orienté, est un triplet $(V,E,\psi)$ où:
\begin{itemize}
\item $V$ est un ensemble dont les éléments sont appelés sommets ou nœuds;
\item $E$ est un ensemble dont les éléments sont appelés arêtes;
\item $\psi$ est une fonction, dite fonction d'incidence, qui associe à chaque arête un couple de sommets. Ici, l'ordre au sein du couple de sommets a de l'importance, il signifie qu'un sommet est le nœud de départ de l'arête, l'autre étant le nœud d'arrivée.\\
\end{itemize}

Un \textbf{parcours} est une suite $v_0e1v_1e_2...e_nv_n$, où $v_1;v_2;...$ sont des sommets, et $e_1;e_2;...$ sont des arêtes. La longueur du parcours est son nombre d'arêtes $n$. Le sommet d'origine est $v_0$, le sommet de destination $v_n$. Les autres sommets sont dits intérieurs. Un parcours est fermé si $v_0 = v_n$.\\

Un \textbf{chemin} est un parcours dont les sommets sont tous distincts.\\

Un \textbf{cycle} est un parcours fermé dont les sommets d'origine et intérieurs sont tous distincts. Un graphe qui ne contient pas de cycle est dit acyclique.
