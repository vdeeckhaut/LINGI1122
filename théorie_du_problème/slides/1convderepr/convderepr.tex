\documentclass[10pt]{beamer}
\usepackage[english]{babel}
\usepackage[utf8]{inputenc}
\usepackage[squaren, Gray]{SIunits}
\usetheme{Warsaw}
\usepackage{wrapfig}
\usepackage{multirow}
\usepackage{blindtext}
\usepackage{tabularx}
\usepackage{color}
\usepackage{circuitikz}
\usepackage[squaren,Gray]{SIunits}
\usepackage{numprint}
\usepackage{qtree}
\usepackage{multicol}
\setbeamertemplate{navigation symbols}{}
\newcommand{\col}[1]{\textcolor{red}{#1}}

\title[Détection de cycle dans un graphe dirigé]{Détection de cycle dans un graphe dirigé}
\subtitle{Conventions de représentations et théorie du problème}
\author{Groupe 9}
\institute{EPL - LINGI1122}
\date{21 février 2014}

\begin{document}
\begin{frame}[allowframebreaks]{Conventions de représentation}
\textbf{\textit{Adjacent List Stucture}}\\
\begin{multicols}{2}
\includegraphics[scale=0.35]{images/schema.jpg}\\
\textbf{Objet noeud} : \begin{itemize}
\item Référence vers une liste chainée contenant des références vers les arêtes sortantes du noeud
\item Entier \textit{incounter} contenant le nombre d'arêtes entrantes au noeud\\
\end{itemize}

\textbf{Objet arête} : \begin{itemize}
\item Référence vers le noeud dont l'arête est la destination
\item Référence vers la collection correspondant à l'arête non nécessaire
\end{itemize}
\end{multicols}

\newpage
Avantages : \\
\begin{itemize}
\item Facilité d'implémentation
\item Complexité de l'algorithme en $\mathcal{O}(n+m)$ avec $n$ le nombre de noeuds et $m$ le nombre d'arêtes. C'est le meilleur choix possible car toutes les arêtes et tous les noeuds sont parcourus dans le pire des cas. De plus, la complexité est meilleure par rapport à
\begin{itemize}
\item l'\textit{edgelist structure} : \textit{AdjacentEdges} en $\mathcal{O}(m)$
\item l'\textit{adjacency matrix structure} : \textit{AdjacentEdges} en $\mathcal{O}(n)$
\end{itemize}
En effet, la complexité de \textit{AdjacentEdges} est dans notre cas de $\mathcal{O}(1)$
\end{itemize}
\end{frame}
\end{document}
